\documentclass[a4paper,12pt]{article}
\usepackage[margin=2.5cm]{geometry}
\usepackage[french]{babel}
\usepackage{amsmath,amssymb}
\usepackage{xcolor}
\usepackage{mdframed}
\usepackage{stmaryrd}

% Define style for correction boxes
\definecolor{corrigecolor}{gray}{0.7}

\newmdenv[
linecolor=corrigecolor,
linewidth=2pt,
leftline=true,
rightline=false,
topline=false,
bottomline=false,
skipabove=6pt,
skipbelow=6pt,
innertopmargin=4pt,
innerbottommargin=4pt,
innerleftmargin=6pt,
innerrightmargin=4pt,
backgroundcolor=gray!5
]{correctionbox}




\title{Corrigé colle S8 \\
	MPI/MPI* du lycée Faidherbe \\
	\large Exercices 16, 17, 18}
\author{Brahim EL HAMDANI}

\begin{document}
	\maketitle
	
	\subsection*{Exercice 16}
    Soient $A, B \in M_n(\mathbb{C})$.
	Montrer que $(A \text{ et } B \text{ ont une valeur propre commune}) \iff (\exists P \neq 0, \text{ tel que } AP = PB)$.
	
	\begin{correctionbox}
		\textbf{Sens direct} : \\
		On suppose que $A$ et $B$ ont une valeur propre commune $\lambda$. \\
		Soit $X \in \mathbb{C}^n$ un vecteur propre de $A$ associé à $\lambda$, et $Y \in \mathbb{C}^n$ un vecteur propre de $B^\top$ associé à $\lambda$ (car $\chi_B = \chi_{B^\top}$, donc $B$ et $B^\top$ ont mêmes valeurs propres). \\
		On pose $P = X Y^\top$. Vérifions que $P$ convient :
		\begin{itemize}
			\item $P \neq 0$ car $X \neq 0$ et $Y \neq 0$ (c'est pas dur à montrer)
			\item $AP = A(X Y^\top) = (AX) Y^\top = (\lambda X) Y^\top = \lambda (X Y^\top)$
			\item $PB = (X Y^\top) B = X (Y^\top B) = X (B^\top Y)^\top = X (\lambda Y)^\top = \lambda (X Y^\top)$
		\end{itemize}
		Ainsi, $AP = \lambda P = PB$, donc $AP = PB$ avec $P \neq 0$.	
		\\\\
		\textbf{Réciproque} : \\
		On suppose qu'il existe $P \neq 0$ tel que $AP = PB$. \\ On montre par récurrence (big flemme) que pour tout $k \in \mathbb{N}$, $A^k P = P B^k$ :\\ Il vient alors que pour tout polynôme $Q \in \mathbb{C}[X]$, on a $Q(A)P = P Q(B)$.\\En particulier, pour $Q = \chi_A$, on a par le théorème de Cayley-Hamilton :
		$$\chi_A(A) P = P \chi_A(B) = O_n$$
		Comme $P \neq 0$, on en déduit que $\chi_A(B)$ n'est pas inversible. \\
		
		Or, sur $\mathbb{C}$, $\chi_A$ est scindé :
		$$\chi_A(X) = \prod_{\lambda \in \mathrm{Sp}(A)} (X - \lambda)^{m_\lambda}$$\\
		
		On a donc :
		$$\chi_A(B) = \prod_{\lambda \in \mathrm{Sp}(A)} (B - \lambda I_n)^{m_\lambda}$$
		Comme $\chi_A(B)$ n'est pas inversible, il existe $\lambda \in \mathrm{Sp}(A)$ tel que $B - \lambda I_n$ n'est pas inversible (sinon chaque $B - \lambda I_n$ serait inversible et leur produit aussi). \\ Ainsi, $\ker(B - \lambda I_n) \neq \{0\}$, donc $\lambda$ est valeur propre de $B$. \\\\Conclusion : $A$ et $B$ ont une valeur propre commune.
	\end{correctionbox}



    \subsection*{Exercice 17}
    Soit \( u \in \mathcal{L}(\mathbb{R}^n) \) tel que : \( \exists q \in \mathbb{N}^* / u^q = \mathrm{Id} \). 
    Montrer que \( \dim (\ker (u - \mathrm{Id})) = \frac{1}{q} \sum_{k=1}^{q} \mathrm{Tr}(u^k) \).
    
    \begin{correctionbox}
        Soit \( A \) la matrice de \( u \) dans une base de \( \mathbb{R}^n \). \\ \\
        On a \( A^q = I_n \), donc le polynôme \( Q(X) = X^q - 1 \) est annulateur de \( A \). \\ \\
        Sur \( \mathbb{C} \), \( Q \) est scindé à racines simples (les racines \( q \)-ièmes de l'unité).Ainsi, \( A \) est diagonalisable sur \( \mathbb{C} \) et \( \mathrm{Sp}(A) \subset \{ \omega^k \mid k \in \llbracket 0, q-1 \rrbracket \} \) où \( \omega = e^{2\pi i/q} \). \\ \\
        Soit \( m(1) = \dim(\ker(u - \mathrm{Id})) \) la multiplicité de la valeur propre 1. On veut montrer que \( m(1) = \frac{1}{q} \sum_{k=1}^{q} \mathrm{Tr}(u^k) \). \\ \\
        Comme \( A \) est diagonalisable, elle est semblable à une matrice diagonale :
        \[
        \Delta = \mathrm{diag}(\lambda_1, \dots, \lambda_n) \quad \text{avec} \quad \lambda_i \in Sp(A)\quad \text{et} \quad \lambda_i^q = 1 \    
        \] \\
        On a alors pour tout \( k \in \mathbb{N}^* \) :
        \[
        \mathrm{Tr}(u^k) = \mathrm{Tr}(A^k) = \mathrm{Tr}(\Delta^k) = \sum_{i=1}^n \lambda_i^k
        \]
        D'où :
        \[
        \sum_{k=1}^{q} \mathrm{Tr}(u^k) = \sum_{k=1}^{q} \sum_{i=1}^n \lambda_i^k 
        = \sum_{i=1}^n \sum_{k=1}^{q} \lambda_i^k
        \]
        Pour chaque \( \lambda_i \), on distingue deux cas :
        \begin{itemize}
        \item Si \( \lambda_i = 1 \), alors \( \sum_{k=1}^{q} \lambda_i^k = \sum_{k=1}^{q} 1 = q \)
        \item Si \( \lambda_i \neq 1 \), alors c'est une racine \( q \)-ième de l'unité différente de 1, donc :
        \[
        \sum_{k=1}^{q} \lambda_i^k = \lambda_i \frac{1 - \lambda_i^q}{1 - \lambda_i} = \lambda_i \frac{1 - 1}{1 - \lambda_i} = 0
        \]
        \end{itemize}
        Il vient alors que :
        \[
        \sum_{k=1}^{q} \mathrm{Tr}(u^k) = m(1) \cdot q
        \]
        On en déduit :
        \[
        \dim(\ker(u - \mathrm{Id})) = m(1) = \frac{1}{q} \sum_{k=1}^{q} \mathrm{Tr}(u^k)
        \]
    \end{correctionbox}



    \subsection*{Exercice 18}
    Soient \( A \) et \( X \) dans \( M_n(\mathbb{R}) \). On suppose que \( X \) est de rang 1. Montrer que \( \det(A + X) \det(A - X) \leq \det(A)^2 \).
    
    \begin{correctionbox}
        Comme \( X \) est de rang 1, il existe \( P, Q \in GL_n(\mathbb{R}) \) telles que \( X = P J_1 Q \) 
        \\ \\
        On rappelle que \( J_1 = \begin{pmatrix} 1 & 0 & \cdots & 0 \\ 0 & 0 & \cdots & 0 \\ \vdots & \vdots & \ddots & \vdots \\ 0 & 0 & \cdots & 0 \end{pmatrix} \). 
        \\ \\ \\
        Considérons le polynôme \( P(t) = \det(A + tX) \). \\
        On a \( P(t) = \det(A + tP J_1 Q) = \det(P(P^{-1}AQ^{-1} + tJ_1)Q) = \det(P)\det(Q)\det(B + tJ_1) \) où \( B = P^{-1}AQ^{-1} \). \\ \\
        Développons \( \det(B + tJ_1) \) par rapport à la première colonne. Notons \( B = (b_{i,j}) \). \\ \\
        \[
        \begin{aligned}
        \det(B + tJ_1) &= \begin{vmatrix}
        b_{1,1} + t & b_{1,2} & \cdots & b_{1,n} \\
        b_{2,1}     & b_{2,2} & \cdots & b_{2,n} \\
        \vdots     & \vdots & \ddots & \vdots \\
        b_{n,1}     & b_{n,2} & \cdots & b_{n,n}
        \end{vmatrix} \\
        &= (b_{1,1} + t) \cdot \Delta_{1,1} + \sum_{k=2}^n (-1)^{k+1} b_{k,1} \cdot \Delta_{k,1}
        \end{aligned}
        \] \\ \\
        où \( \Delta_{k,1} \) est le déterminant de la matrice extraite obtenue en supprimant la k-ième ligne et la première colonne. \\ \\
        On en déduit que P(t) est de degré 1 donc : 
        \[
        \exists (a,b) \in \mathbb{R}^2, P(t) = at + b
        \]
        On a alors :
        \[
        P(1) = \det(A + X) = a + b, \quad P(-1) = \det(A - X) = b - a, \quad P(0) = \det(A) = b
        \]
        Calculons le produit :
        \[
        \det(A + X)\det(A - X) = (a + b)(b - a) = b^2 - a^2 \leq b^2 = \det(A)^2
        \]
    \end{correctionbox}
\end{document}
