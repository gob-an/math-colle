\documentclass[a4paper, 12pt]{article}
\usepackage{colle}

\title{Corrigé colle S19 \\ MPI/MPI* du lycée Faidherbe \\ \large Exercices 15, 20 et 21}
\author{Léane Parent}

\begin{document}
	\maketitle
	
	\subsection*{Exercice 15}
	Soit $p$ un nombre premier. On pose $q = (p^2 - p) (p^2 - 1)$. Soit $A \in M_2(\sfrac \Z {p\Z})$.
	\begin{enumerate}[1)]
		\item Donner le cardinal de $GL_2(\sfrac \Z {p\Z})$.
		\begin{corr}
			Une matrice est inversible ssi la famille de ses coefficients est libre.
			
			On a donc $p^2-1$ choix pour la première colonne (les vecteurs non nuls), et $p^2-p$ choix pour la seconde (les vecteurs non colinéaires au premier)
			
			Soit, par principe multiplicatif: $\#GL_2(\sfrac \Z {p\Z}) = (p^2-p)(p^2-1) = q$.
		\end{corr}
		
		\item Montrer que $A^{q+2} = A^2$
		\begin{corr}
			D'après la question précédente, $GL_2(\sfrac \Z {p\Z})$ est un groupe d'ordre $q$. Ainsi, $o(A)\,|\,q$.
			
			On a donc $A^{q+2} = A^q A^2 = A^2$.
		\end{corr}
		\item Quel est le cardinal de $GL_n(\sfrac \Z {p\Z})$, pour $n\geqslant 1$?
		\begin{corr}
			On montre de même qu'en question (1) que:
			$$\#GL_n(\sfrac \Z {p\Z}) = (p^n-1)(p^n-p)\dots(p^n-p^{n-1})$$.
		\end{corr}
		\item Quel est le cardinal de $SL_n(\sfrac Z {pZ})$?
		\begin{corr}
			$$GL_n(\sfrac \Z {p\Z}) = \bigcup_{i=0}^p \{M \, |\, \det M = i\}$$
			
			Or, si $M$ et $M'$ ont même déterminant, $M^{-1}M' \in SL_n(\sfrac z {pZ})$. On a donc, pour $A$ de déterminant $i$:
			$$ A \cdot SL_n(\sfrac \Z {p\Z}) = \{M \, |\, \det M = i\}$$
			
			On en déduit $\# GL_n(\sfrac \Z {p\Z}) = p\, \# SL_n(\sfrac \Z {p\Z})$, d'où:
			\begin{align*}
				 \#SL_n(\sfrac \Z {p\Z}) & = \frac 1 p (p^n-1)(p^n-p)\dots(p^n-p^{n-1}) \\
				 & = (p^n-1)(p^n-p)\dots(p^n-p^{n-2}) (p^{n-1} - p^{n-2})
			\end{align*}
		\end{corr}
	\end{enumerate}
	
	
	\subsection*{Exercice 20}
	On définit $S_2(\Z)$ l'ensemble des matrices de taille $2\times 2$ à coefficients entiers de déterminant 1.
	On définit également $S = \begin{pmatrix}
		0 & -1 \\ 1 & 0
	\end{pmatrix}$ et $T = \begin{pmatrix}
		1 & 1 \\ 0 & 1
	\end{pmatrix}$.
		
	\begin{enumerate}[1)]
		\item Montrer que $SL_2(\Z)$ est un groupe.
		\begin{corr}
			Montrons qu'il s'agit d'un sous-groupe de $GL_2(\R)$. La stabilité par produit est immédiate.
			
			De plus, si $M \in SL_2(\Z)$, $M$ est inversible, et $M^{-1} = \frac 1 {\det M} \mathrm{Com}\, M = \mathrm{Com}\, M \in SL_2(\Z)$. (Où $\mathrm{Com} \, M$ désigne la comatrice de $M$.)
			
			Il s'agit bien d'un sous-groupe, donc d'un groupe.
		\end{corr}
		
		\item Montrer que $S$ et $T$ engendrent $SL_2(\Z)$.
		\begin{corr}
			D'une part, $S$ et $T$ appartiennent bien à $SL_2(\Z)$
			
			On vérifie par le calcul que, pour $M = \begin{pmatrix}
				a & b \\
				c & d
			\end{pmatrix}$, $k \in \Z$:
			$$ S M = \begin{pmatrix} -c & -d \\ a & b \end{pmatrix} \text{\qquad et \qquad} T^k M = \begin{pmatrix}
					a + kc & b + kd \\ c & d
				\end{pmatrix} $$
				
			On a alors, si $a=cq+r$ est la division  euclidienne de $a$ par $c$:
			$$ S T^{-q} M = \begin{pmatrix}
				-c & * \\ r & *
			\end{pmatrix} $$
			
			On applique alors récursivement ce processus pour obtenir $M'$ de la forme $\begin{pmatrix}
				\alpha & \beta \\ 0 & \gamma
			\end{pmatrix}$ (algorithme d'Euclide).
			
			Or, par produit, $M' \in SL_2(\Z)$. Ainsi, elle est de déterminant 1: on a $\alpha = \gamma = \pm 1$. Quitte à multiplier à gauche par $S^2 = - I_2$, on suppose $\alpha=\gamma=1$.
			
			Mézalor $M' = T^\beta$: en inversant les étapes de l'algorithme, on obtient une décomposition de $M$ selon $S$ et $T$.
			
			On a bien $SL_2(\Z) = \scal S T$.
		\end{corr}
		
		\item Question posée pendant le temps restant: On admet que $A = \begin{pmatrix}
			17 & 29 \\ 7 & 12
		\end{pmatrix}$ est dans $SL_2(\Z)$. Déterminer sa décomposition avec les matrices $S$ et $T$.
		\begin{corr}
			Peut-être que le candidat avait du temps restant, mais moi j'en ai pas avant d'aller me coucher.
			
			Appliquez la démonstration de la question précédente.
		\end{corr}
	\end{enumerate}
	
	\subsection*{Exercice 21}
	Montrer qu'un sous-groupe discret de $\R^n$ admet une $\Z$-base. \\
	
	\textit{Petit point vocabulaire}:
	\begin{itemize}
		\item Un ensemble discret $E$, c'est un ensemble dont tous les points sont isolés, c'est-à-dire que pour tout $x \in E$, il existe $\eps$ tq $B_o(x, \eps) \cap E = \{x\}$ (ie il n'existe pas de point arbitrairement proche de $x$). (La vraie définition est que toute intersection de $E$ et d'un compact est finie, mais celle-ci est équivalente.)
		\item Parler de $\Z$-base, c'est considérer $G$ comme un $\Z$-module: c'est comme un ev, sauf que les scalaires sont dans un anneau (ici $\Z$) et pas forcément dans un corps.
		\\
		\item Par souci de concision, on notera $\Vect A = \Vect[\R] A $ le $\R$-espace engendré par $A$, et $\Vect[\Z] A$ le $Z$-module engendré par $A$.
	\end{itemize}
	\begin{corr}
		On note $G \leqslant \R^n$ discret.

		Construisons notre base.
		\begin{itemize}
			\item On note $B_0 = \emptyset$
			\item Si $x \in G\,\backslash \Vect B_i$ existe, ie si $G\not \subset \Vect B_i$, $\Vect x \cap G$ est discret, donc il existe $x' \in \Vect x \cap G$ non nul de distance minimale (c'est à rédiger, je vous laisse le faire) à $\Vect B_i$. On note $B_{i+1} = B_i \cup {x'}$.
		\end{itemize}
		
		Ce processus finit (car $\Vect B_i$ est de dimension $i$ dans un espace de dimension $n$) et donne donc une $\R$-base de $\Vect G$. On note $B = (x_1,\, \dots \, x_k)$ sa valeur finale. \\
		
		Montrons que $B$ est une $\Z$-base de $G$. Soit $x \in G$, $x= \lambda x_i + x'$, avec $x' \in \Vect B_{k-1}$.
		
		Si $\lambda'$ est la partie fractionnaire de $\lambda$, et $x_k = p+h$ la décomposition de $x_k$ selon $\Vect B_i$ et son orthogonal, on a alors:
		\begin{itemize}
			\item $\|h\|$ est la distance de $x_k$ à $\Vect B_i$.
			\item $\lambda' x_k + x' \in G$ (car $x_i\in G$, d'où le résultat en soustrayant $x_i$ $\lfloor \lambda \rfloor$ fois), de distance à $\Vect B_i$ égale à $\lambda' \|h\|$
		\end{itemize}
		Or, $\lambda' \|h\| < \| h\|$, donc $\lambda' x_k + x' \in G$ est dans $G$ de distance inférieure à la distance minimale: Ainsi, $\lambda' x_k + x' \not\in G \backslash \Vect B_i$, donc dans $B_i$: $\lambda'=0$, soit $\lambda\in\Z$. \\
		
		On applique ensuite récursivement à $x'$, en considérant $G \cap \Vect B_i$ qui est bien un groupe: tous les coefficients sont entiers, donc $B$ est une $\Z$-base de $G$.
	\end{corr}
\end{document}