\documentclass[a4paper, 12pt]{article}
\usepackage{colle}

\title{Corrigé colle S12 \\
	MPI/MPI* du lycée Faidherbe \\
	\large Exercices 20 et 22}
\author{Léane Parent}

\begin{document}
	\maketitle
	
	\subsection*{Exercice 20}
	Montrer que $e$ est irrationnel.
	\begin{correctionbox}
		On suppose $e$ rationnel, on note $e = \frac p q$, avec $p$, $q \in \N$. \\
		Considérons la quantité $pq!$. On a: \\
		\begin{align*}
			 q!p & = q! q \frac p q = q!q e \\
			 & = q!q \sum_{k=0}^{+\infty} \frac 1 {k!} \\
			 & = q \sum_{k=0}^q \frac{q!}{k!} + q \sum_{k=q+1}^{+\infty} \frac{q!}{k!}
		\end{align*}
		Le membre de gauche est évidemment entier. Le second terme également, par somme d'entiers. Notons le second terme $\delta$. \\
		Montrons que $0<\delta<1$. \\
		La première inégalité est évidente. De plus, on a $\displaystyle \delta = q \sum_{k=q+1}^{+\infty} \frac{q!}{k!} = q \sum_{k=1}^{+\infty} \frac{q!}{(q+k)!}$. \\
		Or, pour $k>0$, on a $\frac {q!} {(q+k)!} \leqslant \frac 1 {(q+1)^k}$, avec égalité ssi $k=1$. \\
		On en déduit $\displaystyle \delta = q \sum_{k=1}^{+\infty} \frac{q!}{(q+k)!} < q \sum_{k=1}^{+\infty} \frac 1 {(q+1)^k} = q \frac 1 {q+1} \frac 1 {1 - \frac 1 {q+1}} = \frac q {q+1} \frac {q+1} q = 1 $. \\
		On a bien montré $0<\delta<1$. Or, $\delta \in \Z$, par différence, ce qui est absurde. \\ \\
		Ainsi, $e \not \in \Q$.
	\end{correctionbox}
	
	\subsection*{Exercice 22 \textnormal{(non fini)}}
	On suppose l'existence de $A \subset \N$ tq $\displaystyle \sum_{n \in A} \frac {x^n} {n!} \mathop{\sim}\limits_{x\rightarrow +\infty} \frac {e^x} {x^2} $
	\begin{enumerate}
		\item Pour $I \subset A$ fini, calculer $\displaystyle \int_0^{+\infty} \sum_{n\in I} \frac {x^n e^{-x}} {n!} dx$.
		\begin{correctionbox}
			La somme étant finie, on peut permuter somme et intégrale. \\
			Calculons, par récurrence sur $n$ (eww), $\displaystyle I_n = \int_{0}^{+\infty} x^n e^{-x} dx$. \\
			On a, pour $n>0$, $\displaystyle I_n = \int_{0}^{+\infty} x^n e^{-x} dx = -[x^n e^{-x}]_0^{+\infty} + \int_{0}^{+\infty} x^{n-1} e^{-x} dx = 0 + nI_{n-1}$ (le crochet vaut 0 parce qu'on a supposé $n \not = 0$, il faut faire attention!) Le crochet converge, et $I_{n-1}$ converge par HR, d'où la convergence de $I_n$, et l'égalité. \\
			On en déduit immédiatement $I_n = n!$. \\ \\
			Dès lors, l'intégrale de l'énoncé vaut $\displaystyle \sum_{n\in I} 1 = |I|$. \\
			\\
			Or, on a $\displaystyle \sum_{n \in I} \frac {x^n e^{-x}} {n!} \leqslant \sum_{n \in A} \frac {x^n e^{-x}} {n!}$. De plus, $\displaystyle x \mapsto \sum_{n \in A} \frac {x^n e^{-x}} {n!}$ est continue sur $[0, +\infty[$, positive, et équivalente en $+\infty$ à $x \mapsto \frac 1 {x^2}$, qui est intégrable sur $[0, +\infty[$. Dès lors, $\displaystyle x \mapsto \sum_{n \in A} \frac {x^n e^{-x}} {n!}$ est intégrable, et on en déduit que l'intégrale de l'énoncé est bornée. \\
			\\
			Dès lors, les parties finies de $A$ sont de cardinal bornés, donc $A$ est fini.
			
		\end{correctionbox}
		\item Qu'en conclut-on?
		\begin{correctionbox}
			On a alors $\displaystyle \sum_{n \in A} \frac {x^{n+2}} {n!} \mathop{\sim}\limits_{x\rightarrow +\infty} e^x$, ie un polynôme équivalent à la fonction exponentielle, ce qui est absurde. Un tel $A$ n'existe pas.
		\end{correctionbox}
	\end{enumerate}
	
\end{document}