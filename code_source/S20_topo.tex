\documentclass[a4paper, 12pt]{article}
\usepackage{colle}

\title{Corrigé colle S20 \\ MPI/MPI* du lycée Faidherbe \\ \large{Exercice 18}}
\author{Léane Parent}

\begin{document}
	\maketitle
	Soit $E$ un espace vectoriel normé et $K$ une partie compacte non vide de $E$.
	\begin{enumerate}[a)]
		\item Soit $(L_n)_{n\in\N}$ une suite de fermés non vides inclus dans $K$. Montrer que $\displaystyle \bigcap_{n=0}^{+\infty}L_n \not = \emptyset$.
		\begin{corr}
			On choisit, pour tout $n$, $x_n \in L_n$. On a ainsi construit $(x_n) \in K^\N$.
			
			Or, $K$ étant compact, $(x_n)$ admet une suite extraite convergente. Soit $\vphi$ une extractrice tq $(x_{\vphi(n)})$ CV vers un certain vecteur $x$.
			
			Mq $x \in \displaystyle \bigcap_{n=0}^{+\infty}L_n \not = \emptyset$.
			
			Soit $n \in \N$. Alors, pour tout $k \geqslant n$, $x_k\in L_k \subset L_n$. De plus, $\vphi(k)\geqslant k \geqslant n$ d'où $x_{\vphi(k)} \in L_n$. Dès lors $(x_{\vphi(k)})_{k\geqslant n}$ est une suite d'éléments du fermé $L_n$, qui converge vers $x$, d'où $x\in L_n$ (par caractérisation séquentielle des fermés).
			
			On a bien $x\in L_n$ pour tout $n$, d'où $x \in \displaystyle \bigcap_{n=0}^{+\infty}L_n \not = \emptyset$: cet ensemble n'est pas vide.
		\end{corr}
		
		\item Soit $(f_n)_{n\in\N}$ une suite décroissante de fonctions continues de $K$ dans $\R$ qui converge simplement vers $g: K \rightarrow \R$. Montrer que la convergence est uniforme.
		
		\begin{corr}
			Quitte à considérer $(f_n - f)$, où $(f_n)$ CVS vers $f$, on suppose $(f_n)$ CVS vers la fonction nulle. On remarque que les $(f_n)$ sont positives (car l'inégalité large passe à la limite)
			
			Supposons que $(f_n)$ ne converge pas uniformément. Ainsi, pour $\eps>0$, il existe une extractrice $\vphi$, une suite $x_n$ tq, pour tout $n$, $f_{\vphi(n)}(x_n) > \eps$. De plus, $(f_n)$ est décroissante d'où, pour tout $n$, $k\geqslant n$, $f_{\vphi(n)}(x_\vphi(k)) > \eps$. (De fait, $n \leqslant k \leqslant \vphi(k)$.)
			
			Or, les $(x_k)$ sont dans $K$, qui est compact, donc il existe $x$ valeur d'adhérance de cette suite. Quitte à extraire, on suppose $x_k \cv x$. \\
			
			Or, les $(f_n)$ sont continues par hypothèse, d'où: $f_n(x_{\vphi(k)}) \cv{k\rightarrow +\infty} f_n(x)$, et cette valeur est alors supérieure ou égale à $\eps$ (car l'inégalité large passe à la limite). Dès lors, $f_n(x) \not \rightarrow 0$, ce qui est absurde.
			
			On a ainsi $(f_n)$ CVU vers f.
			
			(Le truc marrant, c'est que ça fonctionne avec les suites décroissantes de fonctions et avec les suites de fonctions décroissantes, on appelle ça les théorèmes de Dini.)
		\end{corr}
	\end{enumerate}
	
	\subsection*{Exercice 19}
	On munit $\C[X]$ de $\| \displaystyle \sum_{k=0}^n a_k X^k \| = \max|a_k|$ (la norme infinie sur $\C[X]$).	Soit $d$ un entier naturel.
	\begin{enumerate}[a)]
		\item Existe-t-il $K_1 \in \R_+$ tq $\forall P, Q \in \C_d[X], \|P\cdot Q\| \leqslant K_1 \|P\|\|Q\|$?
		\begin{corr}
			Soit $P(X) = \alpha_0 + \dots\, \alpha_d X^d$, $Q(X) = \beta_0 + \dots\, \beta_d X^d$.
			
			On a alors le coefficient devant $X^k$ dans $P\cdot Q$ égal à $\displaystyle \sum_{i+j=k} (\alpha_i  \beta_j) = \sum_{j=k-d}^k (\alpha_{k-j} \beta_j)$ (par produit de Cauchy). On a de plus:
			$$ \left| \sum_{j=k-d}^k (\alpha_{k-j} \beta_j) \right| \leqslant \sum_{j=k-d}^k |\alpha_{k-j} \beta_j| \leqslant \sum_{j=k-d}^k \|P\| \|Q\| = (d+1) \|P\|\|Q\|$$
			
			$K_1 = d+1$ convient.
		\end{corr}
		
		\item Existe-t-il $K_2 \in \R_+$ tq $\forall P, Q \in \C[X], \|P\cdot Q\| \leqslant  K_2 \|P\|\|Q\|$?
		\begin{corr}
			On note $P(X) = 1 + X + \dots \, X^n $.
			
			On a alors $\|P\| = 1$. Cependant, le coefficient devant $X^n$ de $P\cdot P$ est $n$. Ainsi, on a un polynôme tq $\|P\cdot P\| \geqslant n \|P\|\|P\|$, d'où, si un tel $K_2$ existait, $K_2 > n$ pour tout $n$, ce qui est absurde. Un tel $K_2$ n'existe pas.
		\end{corr}
		
		\item Existe-t-il $K_3 \in \R_+$ tq $\forall P, Q \in \C_d[X], \|P\|\|Q\| \leqslant  K_3\|P\cdot Q\|$?
		\begin{corr}
			Le cas où l'un des deux polynômes est de norme nulle est immédiat (de fait, ce polynôme serait nul, ce qui permet de conclure).
			
			Dès lors, on considère $\tilde P = \frac 1 {\|P\|} P$, $\tilde Q = \frac 1 {\|Q\|} Q \in S(0, 1)$
			
			La sphère unité est compacte (elle est évidemment bornée, et par continuité de $\|\cdot \|$ est fermée, le tout en dimension finie), d'où, par théorème des bornes atteintes, $P,\, Q \mapsto PQ$ est minorée en norme par un certain $m$ (cette application étant bilinéaire, elle est continue).
			
			On a donc $\|\tilde P \tilde Q \| \geqslant m$, d'où, en multipliant par $\|P\|\|Q\|$:
			$$ \|P \cdot Q\| \geqslant m \|P\|\|Q\|$$
			
			$ K_3 = \frac 1 m$ convient.
		\end{corr}
		
		\item Existe-t-il $K_4 \in \R_+$ tq $\forall P, Q \in \C[X], \|P\|\|Q\| \leqslant K_4\|P\cdot Q\|$?
		\begin{corr}
			
		\end{corr}
	\end{enumerate}
\end{document}