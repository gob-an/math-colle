\documentclass[a4paper, 12pt]{article}
\usepackage{colle}

\title{Corrigé colle S17 \\ MPI/MPI* du lycée Faidherbe \\ \large Exercice 15 et 16}
\author{Léane Parent}

\begin{document}
	\maketitle
	\begin{comment}
		\subsection*{Exercice 13}
		Pour $f \in \mathcal C^0([0,1], \R)$, on pose $\displaystyle \phi(f): x \mapsto \int_0^1 \inf (x,t)f(t) dt$.
		\begin{enumerate}[a)]
			\item Prouver que $\phi \in \mathcal L(\mathcal C^0([0, 1], \R))$.
			\begin{correctionbox}
				Par théorème de continuité (sinon on peut utiliser le théorème fondamental de l'analyse avec la question (b)), $\phi(f) \in \mathcal C^0([0, 1], \R)$. \\
				De plus, la linéarité de $\phi$ découle de celle de l'intégrale: $\phi$ est bien un endo.
			\end{correctionbox}
			
			\item En utilisant la relation de Chasles, déterminer une autre manière d'écrire $\phi(f)$
			\begin{correctionbox}
				On a évidemment: \begin{align*}
					\inf(x, t) & = x & \text{si $x<t$} \\
					& = t & \text{sinon}
				\end{align*}
				Ainsi, en cassant l'intégrale en $x$, on obtient:
				$$ \phi(f)(x) = \int_0^1 \inf(t, x) f(t) dt = \int_0^x t f(t) dt + x \int_x^1 f(t) dt$$
			\end{correctionbox}
			
			\item Déterminer $\ker \phi$ et $ \Img \phi$.
			\begin{correctionbox}
				On sait par théorème fondamental de l'analyse que les deux termes de $\phi(f)$ sont dérivables. \\
				On a donc $\displaystyle \phi(f)'(x) = xf(x) + \int_x^1 f(t)dt - xf(x) = \int_x^1 f(t) dt$. \\
				Dès lors, si $f \in \ker \phi$, en particulier $\phi(f)' = 0$, ie, pour tout $x$, $\displaystyle \int_x^1 f(t) dt = 0$. \\
				On en déduit immédiatement (en redérivant) que $f=0$: On a $\ker \phi = \{0\}$ \\
				\\
				Soit $g\in \Img \phi$. On note $g=\phi(f)$ \\
				On sait déjà que $\Img\phi \subset \Co[2] {[0,1]}{\R}$ (d'après ce qui précède). \\ On a donc: $g''(x) = -f(x)$. \\
				On en déduit donc $g(x) = -\phi(g'')(x) = \displaystyle \int_0 $ % check en 0, 1? 
			\end{correctionbox}
		\end{enumerate}
	\end{comment}
	
	
	\subsection*{Exercice 15}
	Pour $x>0$, on pose $f (x) = \displaystyle \int_0^{+\infty}	e^{-xt} \sin \sqrt t\, dt$.
	
	\begin{enumerate}[a)]
		\item Mq $f$ est de classe $\Co[1]$
		\begin{correctionbox}
			On applique le théorème $\Co[1]$ (quelle surprise!), pour arriver à:
			$$f'(t) = -\int_0^{+\infty} t e^{-xt} \sin \sqrt t\, dt$$
		\end{correctionbox}
		
		\item Donner la limite de $f$ en $+\infty$.
		\begin{correctionbox}
			On applique le théorème de convergence dominée (quelle surprise!), pour arriver à:
			$$f(t) \cv[t\rightarrow +\infty] 0 $$
		\end{correctionbox}
		
		\item Mq $f$ est solution d'une EDL homogène du premier ordre.
		\begin{correctionbox}
			Par le changement de variable $u=\sqrt t$ ($u^2=t, dt = 2u\,du$, et la racine carrée est bien $\Co[1]$ sur l'ouvert, bijective), on obtient:
			$$ f(x) = 2 \int_0^{+\infty} u e^{-xu^2} \sin u \, du \text{\qquad ainsi que \qquad} f'(x) = -2\int_0^{+\infty} u^3 e^{-xu^2} \sin u \, du $$
			
			On a envie de faire apparaître du $u^3$ dans $f$: IPPons donc.
			\begin{align*}
				f(x) & = -2[ue^{-xu^2}\cos u]_0^{+\infty} & - & 2\int_0^{+\infty} (1-2xu^2)e^{-xu^2} \cos u \,du \\
				& = 0 & - & 2\int_0^{+\infty} (1-2xu^2)e^{-xu^2} \cos u \, du \\
				& = -[(1-2xu^3) e^{-xu^2}\sin u]_0^{+\infty} & - & 2\int_0^{+\infty} (-4xu-2xu+4x^2 u^3) e^{-xu^2} \sin u \, du \\
				& = 0 & + & 12xf(x) - 8x^2f'(x)
			\end{align*}
			(On a à chaque fois intégré le cosinus ou le sinus et dérivé le reste.)
			D'où $f'(x)+\frac{12x-1}{8x^2}f(x) = 0$
		\end{correctionbox}
		
		
		\item Exprimez $f$ à l'aide des fonctions usuelles.
		\begin{correctionbox}
			On en déduit donc $f(x)=ke^{-A(x)}$ (avec $k\in\R$), avec $A$ une primitive de $x\mapsto \frac{12x-1}{8x^2}$, $k \in \R$. \\
			On sait que $\frac{12x-1}{8x^2} = \frac 3 {2x}-\frac 1 {8x^2}$. On reconnaît la dérivée de $\frac 3 2 \log|x| + \frac 1 {8x}$. On a donc:
			$$ f(x) = ke^{\frac 3 2 \log|x| + \frac 1 {8x}} = k \frac{e^{-\frac 1 {8x}}}{x\sqrt x} $$
			
			Il s'agit donc de déterminer $k$.
			
			On a donc $f(x) \sim \frac k {x\sqrt x}$ au voisinage de $+\infty$. \\
			De plus, par le changement de variable $u = xt$ (qu'il faut rédiger), on obtient:
			$$ f(x) = \int_0^{+\infty} e^{-xt} \sin\sqrt t\, dt = \frac 1 x \int_0^{+\infty} e^{-u} \sin \sqrt {\frac u x}\, du \sim \frac 1 {x \sqrt x} \int_0^{+\infty} e^{-u} \sqrt u\, du $$
			Or, on reconnaît après changement de variable $v^2 = u$:
			$$ \int_0^{+\infty} e^{-u} \sqrt u\, du = 2 \int_0^{+\infty} v^2 e^{-v^2}\, dv = -[te^{-t^2}]_0^{+\infty} + \int_ 0^{+\infty} e^{-t^2}dt = \frac{\sqrt \pi} 2 $$
			(après IPP, en dérivant $t$ et intégrant $te^{-t^2}$ et en utilisant la valeur de l'intégrale de Gauss)
			On a enfin: \begin{align*}
				f(x) & = k \frac{e^{-\frac 1 {8x}}}{x\sqrt x} \sim \frac k {x\sqrt x} \\
				& \sim \frac {\sqrt \pi} {2x\sqrt x}
			\end{align*}
			D'où l'on identifie $k = \frac {\sqrt \pi} 2$, soit: $$f(x) = \frac{\sqrt \pi e^{-\frac 1 {8x}}}{2x\sqrt x}$$
		\end{correctionbox}
	\end{enumerate}
	
	\subsection*{Exercice 16}
	\textit{Note}: l'examinateur, par pure mesquinerie, note $f(t) = \int h(t, x) dx$. On prendra garde à ne pas confondre paramètre et variable d'intégration. \\
	\begin{enumerate}[a)]
		\item Déterminer $I$ l'ensemble des réels $t$ tq $x \mapsto e^{-xt} \frac {\sin x} x$ est intégrable
		\begin{correctionbox}
			On pose $h(t, x) = e^{-xt} \frac {\sin x} x$.
			
			On supposera (pour la cohérence avec les questions suivantes) qu'il est question d'intégrabilité sur $\R_ +^*$. \\
			
			$h$ est évidemment continue pour $x \in \R_ +^*$. \\
			De plus, on a, d'une part, $h(t, x) \sim_{x\rightarrow 0} 1$: $x \mapsto h(t, x)$ est bien intégrable au voisinage de 0. \\
			D'autre part, $h(t, x) = O_{x\rightarrow +\infty}(\frac {e^{-xt}} x)$, d'où $x \mapsto h(t, x)$ intégrable au voisinage de $+\infty$ si $x>0$. \\
			
			Si $x\leqslant 0$, on a: \begin{align*}
				\int_0^{+\infty} \left|e^{-xt} \frac {\sin x} x\right| dx & \geqslant \int_0^{+\infty} \frac {|\sin x|} x dx \\
				 & \geqslant \sum_{k=0}^{+\infty} \int_{k\pi}^{(k+1)\pi} \frac {|\sin x|} x dx \text{\qquad dans $\overline{\R_+}$}\\
				 & \geqslant \sum_{k=0}^{+\infty} \frac 1 {k\pi} \int_{k\pi}^{(k+1)\pi} |\sin x| dx
			\end{align*}
			Et cette série est divergente. On a bien vérifié que $x\mapsto h(t, x)$ n'est pas intégrable. \\
			
			Ainsi, cette fonction n'est intégrable que pour $t \in \R_+^*$
		\end{correctionbox}
		
		\item Pour $t\in I$, calculer $\displaystyle \int_0^{+\infty} e^{-xt}\frac {\sin x} x dx$.
		\begin{correctionbox}
			Vous n'êtes pas sans savoir qu'une fonction, c'est une primitive de sa dérivée (si, si). \\
			Dérivons donc. \\
			
			On a $\partial_1 h(t, x) = - e^{-xt} \frac {sin x} x$. \\
			On pose $\phi(t) = \begin{dcases}
				1 & \text{si $x<1$} \\
				e^{-xt} & \text{sinon}
			\end{dcases}$ \\
			On vérifie que $\phi$ est intégrable et domine $\partial_1 h(t, x)$. \\
			Dès lors, par théorème $\Co[1]$ pour les intégrales à paramètre, $f$ est $\Co[1]$, et on a $f(t)= \displaystyle -\int_0^{+\infty} e^{-xt} \sin x dx$ \\
			Calculons cette intégrale. On a $sin x = \Im e^{ix}$. \\
			On a donc par $\R$-linéarité de la partie imaginaire: \\
			$$ f'(t) = -\int_0^{+\infty} e^{-xt} \sin x dx = -\Im \int_0^{+\infty} e^{-xt}e^{it} dx = \Im \frac 1 {i-t} [e^{x(i-t)}]_ {+\infty}^0 = \Im \frac 1 {i-t}$$
			Or, $\frac 1 {i-t} = \frac {-i-t} {1+t^2}$, d'où par passage à la partie imaginaire, $f'(t) = -\frac 1 {1+t^2}$. \\
			On en déduit $f(t) = -\atan t + k$, avec $k$ une constante réelle. \\
			Or, par théorème de convergence dominée (j'en rédige déjà un en-dessous, faudrait pas que je me fatigue), on montre que $f(t) \xrightarrow[t\rightarrow +\infty]{} 0$, d'où $k=\frac\pi 2$. \\
			On a ainsi $f(t) = -\atan t + \frac\pi 2$.
		\end{correctionbox}
		
		\item En déduire $\displaystyle \int_0^{+\infty} \frac{\sin x} x dx$.
		\begin{correctionbox}
			On applique le théorème de convergence dominée: \\
			$x\mapsto h(t, x)$ est dominée (vérifiez-le!) par $x \mapsto \begin{dcases}
				1 & \text{si $x<1$} \\
				\frac {\sin x}{x^2} & \text{sinon}
			\end{dcases}$. \\
			Cette fonction est bien intégrable car en $O(\frac 1 {x^2})$, et $h(x, t)$ est bien continue, d'où: \\
			$f(t) \cv[t\rightarrow 0] \displaystyle \int_0^{+\infty} \frac{\sin x} x dx$. Or, $f(t) \cv \frac \pi 2$, d'où: \\
			
			$$\int_0^{+\infty} \frac{\sin x} x dx = \frac \pi 2$$
		\end{correctionbox}
	\end{enumerate}
\end{document}