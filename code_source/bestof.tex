\documentclass[a4paper, 12pt]{article}

\usepackage[utf8]{inputenc}
\usepackage{xcolor}
\usepackage{mdframed}
\usepackage{lipsum}
\usepackage{colle}
\usepackage{amsthm}

\definecolor{bg}{RGB}{255,235,235} 
\definecolor{line}{RGB}{128,0,32}   

\makeatletter
\newmdenv[
linecolor=line,
linewidth=2pt,
leftline=true,
rightline=false,
topline=false,
bottomline=false,
skipabove=6pt,
skipbelow=6pt,
innertopmargin=2pt,
innerbottommargin=2pt,
innerleftmargin=6pt,
innerrightmargin=4pt,
backgroundcolor=bg,
leftmargin=\dimexpr-\@totalleftmargin\relax,
rightmargin=0pt
]{thbox}
\makeatother

\newenvironment{thm}[1]{%
	\par\medskip
	\noindent\textbf{#1}\par\vspace{-1mm}%
	\begin{thbox}%
	}{%
	\end{thbox}%
	\medskip
}


\title{Best-of}
\author{Léane Parent}

\begin{document}
	\maketitle
	Ce document est dédié aux bêtises que j'ai écrit dans mes corrections (et publié!) (ce qui perd un peu de l'intérêt d'une correction, vous me direz...) \\
	L'auteure décline toute responsabilité en cas d'utilisation de ces propositions dans une copie.
	\\
	\begin{thm}{Premier théorème de goban}
		Si $M\in T^+(\R)$, alors la dimension du noyau de $M$ égale le nombre de 0 sur sa diagonale.
	\end{thm}
	\begin{proof}
		On procède par récurrence sur le nombre de 0 sur la diagonale. \\
		\begin{itemize}
			\item Si les coefficients diagonaux sont non nuls, on sait que la matrice est inversible, d'où le résultat
			
			\item supposons la propriété vraie pour $k$ zéros sur la diagonale. On considère alors une matrice $M$ à $k+1$ zéros sur sa diagonale. Quitte à prendre une matrice semblable (ce qui ne change pas la dimension du noyau), on suppose que $M_{1,1}=0$.
			On écrit alors $M = \begin{pmatrix}
				0 & v \\
				(0) & M'
			\end{pmatrix}$, avec $M'$ de noyau de dimension $k$, par hypothèse de récurrence. \\ \\
			Pour $X \in M_{n,1}(\C)$, on écrit $X = {}^T(x\ X')$. \\
			On a $X\in \ker M$ ssi $vX = 0$ et $M'X = 0$, ie $X' \in \ker M'$. \\
			On va lâchement ignorer la première condition. \\
			On obtient donc $\ker M = \Vect({}^T (1 \ 0\ \dots\ 0), {}^T (0\ X'))_{X'\in \ker M'}$, d'où le résultat
		\end{itemize}
		D'après le principe de récurrence, la proposition est bien vraie pour tout $k$.
	\end{proof}
	\hfill \\
	\textbf{Corollaire 1}: Une matrice nilpotente est nulle. \\
	(\textit{Note}: ce résultat étant connu, on pouvait directement utiliser celui-ci pour démontrer le théorème, en notant, à similitude près, $M = \begin{pmatrix}
		N & * \\
		(0) & M'
	\end{pmatrix}$, avec $N$ nilpotente et $M$ inversible.) \\
	\textbf{Corollaire 2}: Le noyau d'une matrice itérée égale le noyau de la matrice elle-même. \\
	(\textit{Élément de preuve}: S'obtient en trigonalisant dans $M_n(\C)$) \\
	(utilisé dans l'exercice 19 de la semaine 7)
	\\ \\ \\
	\begin{thm}{Second théorème de goban}
		Pour $E$, $F_1$, $F_2$ trois sous-espaces vectoriels du même espace, on a:
		$$ E \cap (F_1 + F_2) = (F_1 \cap E) + (F_2 \cap E)$$
		Autrement dit, l'intersection passe à la somme.
	\end{thm}
	\begin{proof}
		L'inclusion directe est immédiate, et je ne vous ferai pas l'affront de vous en faire la démonstration.
		
		On suppose $x=x_1+x_2$, avec $x_1 \in F_1 \cap E$, $x_2 \in F_2 \cap E$.
		Alors $x_1 + x_2 \in F_1 + F_2$ par définition de la somme, et $x_1 + x_2 \in E$ par stabilité pour $+$, d'où l'inclusion réciproque.
	\end{proof}
	\textbf{Corollaire 1}: Ce résultat reste vrai pour une somme directe. \\
	\textbf{Corollaire 2}: Ce résultat reste vrai pour une somme de $k$ sev. \\
	(Très utile pour décomposer un sous-espace $F$ de $E$ (par exemple en sous-espaces propres de F, etc.), si on a une décomposition de $E$ tout entier (utilisé dans l'exercice 16 de la semaine 7.))
	\\ \\ \\
		\begin{thm}{Troisième théorème de goban}
		Si un ensemble a un minimum inférieur à $x$, alors ce minimum est $x$.
	\end{thm}
	\begin{proof}
		On connaît le théorème (classique!) affirmant qu'un minorant est également minimum. Celui-ci est malheureusement faux. Ainsi, dans le cas d'un ensemble admettant un minimum, $x \leqslant \min E \not \Rightarrow x = \min E$. \\
		Par conséquent, on a bien $x\geqslant \min E \Rightarrow x = \min E$. (Si $P\not \Rightarrow Q$, alors $\neg P \Rightarrow Q$.)
	\end{proof}
	\textbf{Remarque 1}: Ce résultat peut sûrement se généraliser à une borne inférieure. \\
	\textbf{Remarque 2}: Cela démontre la rationnalité de $\pi$. En effet, celui-ci est supérieur à $\min\N$, donc $\min\N=\pi$, et un minimum est atteint donc $\pi\in\N$. En particulier, $\pi\in\Q$. \\ \\
	Ce résultat est très utile pour passer d'une inégalité à une égalité (en prenant $x=\min E$) (utilisé dans l'exercice 26 de la semaine 15).
\end{document}