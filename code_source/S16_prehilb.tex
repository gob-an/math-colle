\documentclass[a4paper, 12pt]{article}
\usepackage{colle}

\title{Corrigé colle S16 \\ MPI/MPI* du lycée Faidherbe \\ \large Exercice 24}
\author{Léane Parent}

\begin{document}
	\maketitle
	
	\subsection*{Exercice 24}
	Soit $E$ euclidien de dimension $n$. \\
	Soient $x_1, \ \dots\ x_k$ tq, pour $i \not = j$, $\langle x_i, x_j\rangle <0$. Montrer que $k$ ne peut pas être trop grand et trouver cette limite. \\
	(On appellera $(x_1,\ \dots\ x_k)$ une famille obstusangle.) 
	\begin{correctionbox}
		Montrons tout d'abord (par récurrence, vous savez que j'adore ça) qu'il existe une famille obtusangle à $n+1$ éléments
		\begin{itemize}
			\item Si $n=1$, il existe $x$ tq $E=\Vect\ x$. Alors $(x, -x)$ convient.
			\item Soit $n\geqslant 1$. Si $\dim E = n+1$, supposons que pour $F$ sev de $E$ de dimension n, $F$ contient une famille obtusangle à $n+1$ éléments. \\
			On prend $x \not = 0$ dans $E$. Soit $F = \Vect\ x ^\perp$. On considère $(x_1, \ \dots \ x_{n+1})$ famille obtusangle de $F$ (qui existe par hypothèse de récurrence).  \\
			On note, pour $\lambda>0$ $f_\lambda (y) = y - \lambda x$. \\
			D'une part, $\langle x, f_\lambda(x_i) \rangle = -\lambda <0$. De plus, $\langle f_\lambda(x_i), f_\lambda(x_j) = \langle x_i, x_j \rangle + \lambda^2 \|x\|^2$. Il existe donc $\lambda$ assez petit pour que cette quantité soit négative pour tous $i$, $j$ ($\lambda = -\frac 1 2 \max\langle x_i, x_j \rangle$ convient.) \\
			$(x, x_1,\ \dots,\ x_{n+1})$ est bien obtusangle, ce qui clôt la récurrence.
		\end{itemize}
		\hfill \\
		Montrons maintenant que cette famille est bien la plus grande.
		Montrons d'abord le lemme suivant: si $(x_i)_i$ est obtusangle, et que $\displaystyle \sum_i \lambda_i x_i$ est une combinaison linéaire non triviale annulant les $(x_i)$, alors les $(\lambda_i)$ sont non nuls et de même signe. \\ \\
		On note $I = \left\{i\ |\ \lambda_i <0\right\}$, $J = \left\{i\ |\ \lambda_i \geqslant 0\right\}$ \\
		Puisque la combinaison linéaire est non triviale, il existe $i$ tq $lambda_i \not = 0$. Quitte à prendre l'opposé de la combinaison linéaire, on peut supposer $\lambda_i > 0$, ie $I \not = \emptyset$ \\
		On a alors: $$\sum_{i\in I} \lambda_i x_i = -\sum_{j\in J} \lambda_j x_j$$
		En notant $S$ la somme de droite, et en prenant le produit scalaire par $S$, on obtient:
		$$\sum_{i,j} \lambda_i \lambda_j \langle x_i,  x_j \rangle = -\|S\|^2 $$
		Or, tous les termes de la somme sont positifs (car les $\lambda_i$ sont positifs, les $\lambda_j$ négatifs, et les $\langle x_i, x_j \rangle$ négatifs car $I$ et $J$ sont disjoints, donc la somme est positive. De plus, le second terme est négatif, donc tous les termes de la somme sont nuls. \\
		Ainsi, tous les $\lambda_j$ sont nuls (puisque $I$ est non vide) \\		
		Ainsi, tous les $\lambda_i$ sont positifs. \\
		De plus, si $\lambda_i=0$, on a:
		$$ 0 = \langle 0, e_i \rangle = \langle \sum_j \lambda_j e_j, e_i \rangle = \sum_j \lambda_j \langle e_i, e_j \rangle $$
		Cette somme est strictement négative car tous ses termes (non nuls, qui existent bel et bien) le sont, ce qui est absurde. Ceci achève la démonstration du lemme.
		\\ \\
		On suppose que l'on a une famille obtusangle $(x_1,\ \dots\ x_{n+2})$ à $n+2$ éléments. \\
		Puisque $E$ est de dimension $n$, il existe $\lambda_1,\ \dots\ \lambda_{n+1}$ tq $\lambda_1 x_1 + \dots + \lambda_n x_n + \lambda_{n+1} (x_{n+1}-x_{n+2}) = 0$. \\
		Ainsi, $\lambda_{n+1}$ et $-\lambda_{n+1}$ sont tous deux coefficients dans une CL annulant une famille obtusangle: ils non nuls et de même signe, ce qui est absurde: une famille obtusangle à $n+2$ éléments n'existe pas.
	\end{correctionbox}
\end{document}