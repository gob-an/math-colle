\documentclass[a4paper, 12pt]{article}

\usepackage{colle}

\title{Corrigé colle S15 \\ MPI/MPI* du lycée Faidherbe \\ \large Exercices 26, 27 }
\author{Léane Parent}

\begin{document}
	\maketitle
	
	\subsection*{Exercice 26}
	Soit $N \geqslant 1$. Soit $(X_n)_{n\in\N^*}$ iid de loi $\mu$ sur $[\![1, N]\!]$ tq $\mu(1)>0$. \\
	On pose $S_n = \displaystyle \sum_{k=0}^n X_k$ (avec $S_0=0$), et $E = \{S_i \ |\ i\in\N\}$. \\
	\begin{enumerate}
		\item Montrer que pour $n\geqslant1$, $\pr(n\in E)=\displaystyle \sum_{k=1}^N \mu(k) \pr(n-k \in E)$
		\begin{correctionbox}
			$(n\in E) = \displaystyle \biguplus_{k=1}^N (X_1=k) \cap (n-k \in E')$, où $E' = \left\{S_i'\ |\ i \in \N^*\backslash\{1\}\right\}$, et $S_i'=\displaystyle \sum_{k=2}^i X_k$. \\
			Or, $S_i' \sim S_{i-1}$, d'où $E'\sim E$. \\
			On a donc $\pr(n\in E) = \displaystyle \sum_{k=1}^N \mu(k) \pr(n-k \in E') = \sum_{k=1}^N \mu(k) \pr\left(n-k \in E\right)$ \\
			\textit{Note}: on peut plus simplement (mais j'ai la flemme de rédiger une deuxième fois) stratifier selon la valeur du dernier élément, et non du premier, ce qui permet d'éviter de justifier les équivalences			
		\end{correctionbox}
		\item Soient $F(z) = \displaystyle \sum_{n \in \N^*} \pr(n\in E) z^n$, $\displaystyle G(z) = \sum_{k=1}^N \mu(k)z^k$. Montrer que $F = \frac 1 {1-G}$.
		\begin{correctionbox}
			On prend $z$ dans le rayon de convergence de $F$
			\begin{align*}
				F(z) & = \sum_{n\in\N^*} \pr(n\in E) z^n = \sum_{n\in\N^*} z^n \sum_{k=1}^N \mu(k) \pr(n-k \in E) \\
				& = \sum_{i\in \N} \sum_{n-k=i} \mu(k) \pr(i\in E) z^n \	\ \text{(tout est sommable dans le rayon de convergence)} \\
				& = \sum_{i\in \N} \sum_{k=1}^N z^i z^k \mu(k) \pr(i\in E) \	\ \text{(les termes pour $k>N$ étant nuls)} \\
				& = \sum_{i\in\N} z^i \pr(i\in E) \sum_{k=1}^N \mu(k) z^k = \sum_{i\in\N} z^i \pr(i\in E) G(z) \\
				& = G(z) (1 + F(z)) \ \ \ \ \ \ \ \ \ \ \ \ \ \ \ \text{(car $\pr(0 \in E)=1$: $S_0=0$)} %c'est très moche...
			\end{align*}
			On a donc $F(z) = G(z) (1 + F(z))$, d'où le résultat.
		\end{correctionbox}
		\item Montrer que $F$ admet un pôle simple en 1, et que les autres pôles sont de module strictement plus grand que 1.
		\begin{correctionbox}
			$F$ admet un pôle en $z$ ssi $G(z)=1$.
			Or, $G(1)=\displaystyle \sum_{k=1}^N \mu(k) = 1$: 1 est pôle de F. \\
			De plus, $G'(1) = \displaystyle \sum_{k=1}^N k \mu(k)$, qui est différent de $0$ car somme de termes positifs dont le premier est non nul. Ainsi, 1 est racine simple de $1-G$, donc 1 est pôle simple. \\ \\
			De plus, si $z$ est pôle de $F$ de module 1, alors $\displaystyle \left|\sum_{k=1}^N \mu(k) z^k\right| \leqslant \sum_{k=1}^N |\mu(k) z^k| \leqslant \sum_{k=1}^N \mu(k) = 1$, et il y a égalité entre le premier et le dernier terme donc égalité à chaque inégalité. Ainsi, (par la première inégalité) les $z^k$ sont positivement colinéaires donc $z\in \R_+$, et (par la seconde inégalité) les $z^k$ sont de module 1: on a bien $z=1$.
		\end{correctionbox}
	\end{enumerate}
	
	
	
	\subsection*{Exercice 27}
	On tire $\sigma \in \perm_n$ aléatoirement. On note $X_n$ la variable aléatoire qui donne le nombre de cycles de $\sigma$.
	\begin{enumerate}
		\item Calculer $\pr(X_n=1)$, $\pr(X_n=n)$.
		\begin{correctionbox}
			On se place dans un cadre d'équiprobabilité. \\
			Si $\sigma \in \perm_n$ contient $n$ cycles, alors $\sigma = \Id$ (car tout point est point fixe: en effet, tout point appartient à un cycle de longueur 1). \\
			Dès lors, $\pr(X_n=n)= \frac 1 {n!}$. \\
			\\
			De plus, si $\sigma$ comporte exactement 1 cycle, alors il s'agit d'un $n$-cycle. \\
			Or, un $n$-cycle dans $\perm_n$ est entièrement déterminé par:
			\begin{itemize}
				\item $\sigma(1)$: $n-1$ choix, car 1 n'est pas point fixe
				\item $\sigma^2(1)$: $n-2$ choix, car $\sigma(1)$ n'est pas point fixe, et ne peut être envoyé sur 1 sous réserve de former un cycle de longueur 2
				\item \dots
				\item $\sigma^n (1) = 1$ pour former un $n$-cycle: 1 choix
			\end{itemize}
			Ainsi, il existe $(n-1)!$ $n$-cycles dans $\perm_n$, d'où $\pr(X_n=1) = \frac {(n-1)!} {n!} = \frac 1 n$.
		\end{correctionbox}
		
		\item Déterminer la fonction génératrice de $X_n$.
		\begin{correctionbox}
			Pour $\sigma \in \perm_{n+1}$, on considère $\sigma' = (\sigma(n+1) \ n+1) \sigma \in \perm_n$ (l'idée c'est qu'il s'agit de $\sigma$ "corrigée" pour que $n+1$ soit point fixe.) \\
			Dès lors, le nombre de cycle de $\sigma$ est, si on note $k$ le nombre de cycles de $\sigma'$ (dans $\perm_n$, c'est-à-dire sans compter le cycle (de longueur 1) de $n+1$):
			 \begin{itemize}
				\item $k+1$ si $n+1$ est point fixe (avec probabilité $\frac 1 {n+1}$, j'ai la flemme de le montrer (ça découle des $n+1$ possibilités pour $\sigma(n+1)$), indépendante de $\sigma'$)
				\item $k$ sinon
			\end{itemize}
			\hfill \\ Montrons maintenant que $\sigma'$ suit une loi uniforme dans $\perm_n$. (flemme $\rightarrow$ plus tard) \\
			on a alors $X_{n+1} \sim X_n + Y_n$, avec $X_n \Vbar Y_n \hookrightarrow \mathcal B (\frac 1 {n+1})$. \\
			Dès lors, en notant $G_n$ la fonction génératrice de $X_n$, on a $G_{n+1}(t) = \frac {n-1+t} n G_n(t)$. On arrive alors à $G_n(t) = \frac 1 {n!} t(t+1)\dots(t+n-1)$.
		\end{correctionbox}
		\item Déterminer son espérance et sa variance
		\begin{correctionbox}
			On a, par linéarité de l'espérance, $\esp(X_{n+1}) = \esp(X_n) + \frac 1 {n+1}$, d'où, par une récurrence immédiate ($X_1=1$), $\esp(X_n) = H_n$, où $H_n$ est la somme partielle de la série harmonique. (On remarque que $\esp(X_n)\sim \log n$.) \\
			\\ De plus, on sait que $X_n \Vbar Y_n$ d'où $\mathds V(X_{n+1})=\mathds V(X_n) + \mathds V (Y_n) = \mathds V(X_n) + \frac n {(n+1)^2}.$ \\
			On en déduit que $\mathds V(X_n) = \displaystyle \sum_{k=1}^n \frac {k-1} {k^2}$ (on vérifie que cette formule est bien vraie en $n=1$) \\
			Ainsi, $\mathds V(X_n) = H_n-  S_n$, avec $S_n = \displaystyle \sum_{k=1}^n \frac 1 {k^2}$. (On remarque que $\mathds V(X_n) \sim \log n$.)
		\end{correctionbox}
	\end{enumerate}
\end{document}