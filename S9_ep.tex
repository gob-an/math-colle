\documentclass[a4paper, 12pt]{article}
\usepackage{colle}

\title{Corrigé colle S9 \\
	MPI/MPI* du lycée Faidherbe \\
	\large Exercices 16 et 17}
\author{Brahim El Hamdani et Léane Parent}

\begin{document}

\maketitle

    \subsection*{Exercice 16 (D'après l'exercice 4 du document "Tribu infinie non dénombrable")}

    
    
    Soit $\Omega$ un ensemble non-vide et $\mathscr{T}$ une tribu sur $\Omega$. Pour $x \in \Omega$, on définit
    \[
    \mathcal{F}_x := \{ T \in \mathscr{T} : x \in T \} \quad \text{et} \quad [x] := \bigcap_{F \in \mathcal{F}_x} F.
    \]
    
    \begin{enumerate}
    \item Montrer que $x \notin [y]$ implique $y \notin [x]$. En déduire que $x \sim y$ si $x \in [y]$ est une relation d'équivalence sur $\Omega$.
    
    \begin{correctionbox}
        Si $x \notin [y]$, alors il existe $A \in \mathscr{T}$ tel que $x \notin A$ et $y \in A$. \\
        Alors $\overline{A} \in \mathscr{T}$, avec $y \notin \overline{A}$ et $x \in \overline{A}$, donc $y \notin [x]$.
        
        Définissons $x \sim y$ par $x \in [y]$. Vérifions que $\sim$ est une relation d'équivalence :
        \begin{itemize}
            \item \textbf{Réflexivité :} $x \in [x]$ car $x \in \bigcap_{F \in \mathcal{F}_x} F$.
            \item \textbf{Symétrie :} Si $x \in [y]$, supposons $y \notin [x]$. Alors $x \notin [y]$ d'après l'implication prouvée, contradiction. Donc $y \in [x]$.
            \item \textbf{Transitivité :} Si $x \in [y]$ et $y \in [z]$, alors pour tout $F \in \mathscr{T}$ avec $z \in F$, on a $y \in F$ (car $y \in [z]$), donc $x \in F$ (car $x \in [y]$). Ainsi $x \in [z]$.
        \end{itemize}
        Donc $\sim$ est une relation d'équivalence.
    \end{correctionbox}
    
    \item Montrer que $\mathcal{P} = \{[x] : x \in \Omega\}$ est une partition de $\Omega$.
    
    \begin{correctionbox}
    \begin{itemize}
            \item Chaque $[x]$ est non vide car $x \in [x]$.
            \item $\bigcup_{x \in \Omega} [x] = \Omega$ car $x \in [x]$.
            \item Si $[x] \cap [y] \neq \emptyset$, soit $z \in [x] \cap [y]$. Alors $z \sim x$ et $z \sim y$, donc $x \sim y$ par symétrie et transitivité, d'où $[x] = [y]$.
        \end{itemize}
        Ainsi $\mathcal{P}$ est une partition de $\Omega$.
    \end{correctionbox}
    
    \item Soit $T \in \mathscr{T}$. Montrer que $T = \bigcup_{x \in T} [x]$.
    
    \begin{correctionbox}
    \begin{itemize}
            \item Pour tout $x \in T$, on a $T \in \mathcal{F}_x$, donc $[x] = \bigcap_{F \in \mathcal{F}_x} F \subseteq T$. Ainsi $\bigcup_{x \in T} [x] \subseteq T$.
            \item Inversement, si $x \in T$, alors $x \in [x]$, donc $T \subseteq \bigcup_{x \in T} [x]$.
        \end{itemize}
        D'où $T = \bigcup_{x \in T} [x]$.
    \end{correctionbox}
    
    \item Montrer que $\mathcal{P}$ est infinie si $\mathscr{T}$ est infinie.
    
    \begin{correctionbox}
        Supposons $\mathcal{P}$ fini, de cardinal $n$. \\
        D'après l'exercice 1, la tribu engendrée par une partition finie de $n$ ensembles non vides disjoints a pour cardinal $2^n$. \\
        Or, par (c), tout $T \in \mathscr{T}$ est réunion de classes de $\mathcal{P}$, donc $\mathscr{T} \subset \sigma(\mathcal{P})$. \\
        Ainsi $|\mathscr{T}| \le 2^n < \infty$. \\
        Par contraposée, si $\mathscr{T}$ est infinie, alors $\mathcal{P}$ est infinie.
    \end{correctionbox}
    
    \item Soit $\mathcal{P}$ infinie. Montrer (par l'absurde) que $\mathscr{T}$ est indénombrable.
    \begin{enumerate}
        \item Montrer que si $\mathscr{T}$ est dénombrable, alors $[x] \in \mathscr{T}$ pour tout $x \in \Omega$. En déduire que $\mathcal{P}$ est dénombrable.
        
        \begin{correctionbox}
            Pour tout $x$, $\mathcal{F}_x = \{ T \in \mathscr{T} : x \in T \}$ est dénombrable. \\
            Alors $[x] = \bigcap_{F \in \mathcal{F}_x} F$ est une intersection dénombrable d'éléments de $\mathscr{T}$, donc $[x] \in \mathscr{T}$. \\
            Ainsi $\mathcal{P} \subset \mathscr{T}$, donc $\mathcal{P}$ est dénombrable.
        \end{correctionbox}
        
        \item Soit $(P_n)_{n \geq 0}$ une énumération de $\mathcal{P}$. S'en servir pour construire une bijection $\phi : \mathscr{P}(\mathbb{N}) \rightarrow \sigma(\{P_n : n \in \mathbb{N}\}) \subset \mathscr{T}$. Conclure.
        
        \begin{correctionbox}
            Puisque $\mathcal{P}$ est dénombrable infinie, on peut indexer ses éléments par $\mathbb{N}$ : soit $f : \mathbb{N} \to \mathcal{P}$ une bijection. On note $P_n = f(n)$. \\
            Les $P_n$ sont disjoints car $\mathcal{P}$ est une partition.\\\\
            Soit $\sigma(\{P_n : n \in \mathbb{N}\})$ la plus petite tribu contenant tous les $P_n$ (on l'appelle la tribu engendrée par $\{P_n : n \in \mathbb{N}\}$ ). \\\\
            Soit $\mathcal{A} = \left\{ \bigcup_{k \in S} P_k : S \subset \mathbb{N} \right\}$. \\
            Montrons que $\sigma(\{P_n\}) = \mathcal{A}$ par double inclusion :
            \begin{itemize}
                \item $\mathcal{A} \subset \sigma(\{P_n\})$ : \\
                      Soit $A \in \mathcal{A}$, donc $A = \bigcup_{k \in S} P_k$ pour un $S \subset \mathbb{N}$. \\
                      Comme chaque $P_k \in \sigma(\{P_n\})$ et $\sigma(\{P_n\})$ est stable par réunion dénombrable, $A \in \sigma(\{P_n\})$.
                \item $\sigma(\{P_n\}) \subset \mathcal{A}$ : \\
                      Vérifions que $\mathcal{A}$ est une tribu :
                      \begin{itemize}
                          \item $\Omega = \bigcup_{k \in \mathbb{N}} P_k \in \mathcal{A}$.
                          \item Si $A = \bigcup_{k \in S} P_k \in \mathcal{A}$, alors $\overline{A} = \bigcup_{k \notin S} P_k \in \mathcal{A}$.
                          \item Si $A_n = \bigcup_{k \in S_n} P_k \in \mathcal{A}$, alors $\bigcup_n A_n = \bigcup_{k \in \bigcup_n S_n} P_k \in \mathcal{A}$.
                        \end{itemize}
                    Donc $\mathcal{A}$ est une tribu contenant tous les $P_n$, donc $\sigma(\{P_n\}) \subset \mathcal{A}$\\
            \end{itemize}
            Définissons $\phi : \mathcal{P}(\mathbb{N}) \to \mathscr{T}$ par : pour tout $S \subset \mathbb{N}$,
            \[
            \phi(S) = \bigcup_{k \in S} P_k.
            \]
            Montrons que $\phi$ est injective. \\
            Soient $S, S' \subset \mathbb{N}$ avec $S \neq S'$. Alors il existe $k_0$ tel que $k_0 \in S \setminus S'$. Puisque les $P_k$ sont disjoints, $P_{k_0} \subset \phi(S)$ mais $P_{k_0} \cap \phi(S') = \emptyset$.Donc $\phi(S) \neq \phi(S')$.\\\\
            Ainsi il existe une injection d'un ensemble non dénombrable dans $\mathscr{T}$ donc $\mathscr{T}$ non dénombrable. Absurde. Donc $\mathscr{T}$ est non dénombrable.
        \end{correctionbox}
    \end{enumerate}
    \end{enumerate}
    
    
    
    
    \subsection*{Exercice 16 \textnormal{(version Léane)}}
    Montrer qu'une tribu infinie est indénombrable.
    
	J'aime pas trop la correction de Brahim, donc je fais la mienne (qui est cependant similaire).
	
	On se donne $T$ une tribu sur $E$.
	\begin{correctionbox}	
		On note $A(x) = \bigcap\limits_{x \in A \in T} A$. \\
		
		Montrons que les $A(x)$ forment une partition de $E$. \\
		Puisque $x \in A(x)$, il suffit de mq les $A(x)$ sont des classes d'équivalence. \\
		Si $y \in A(x)$, mq $x \in A(y)$. Supposons que ce ne soit pas le cas. \\
		Alors il existe $A \in T$ tq $y \in A, x \not \in A$. On a $^C A \in T$, $x \in {}^C A$, $y \not \in {}^C A$. Ainsi, $y \not \in A(x)$, ce qui est absurde. On a bien $A(x) = A(y)$. \\ 
		
		Montrons que, pour $A \in T$, $A \bigcup\limits_{x\in A} A(x)$. \\
		L'inclusion directe et immédiate. De plus, si $x \in A$, $y \in A(x)$, on sait que $y\in A$ pour tout $A$ tq $x \in A$, d'où l'inclusion réciproque. \\
		
		On considère $(x_i)_{i\in I}$ une famille de représentants de classes d'équivalence distinctes. \\
		Soit $\varphi : \begin{dcases}
			\mathcal{P}(I) & \rightarrow T \\
			J & \mapsto \bigcup\limits_{j \in J} A(x_j)
		\end{dcases}$. \\
		
		Puisque les $A(x_j)$ sont disjoints, cette application est injective. \\
		De plus, on a montré plus haut que tout élément de $T$ se décomposait en union de $A(x)$, donc de $A(x_i)$, d'où la surjectivité \\
		
		On suppose $T$ infinie. \\
		Alors, par bijectivité de $\varphi$, $\mathcal P (I)$ est infini donc $I$ l'est également. \\
		On en déduit que $\mathcal P (I)$ est indénombrable (l'ensemble des parties d'un ensemble infini est indénombrable), d'où l'indénombrabilité de $T$. \\
		
		\textit{Note: on a prouvé que si $T$ était finie, alors elle est de cardinal $2^n$ avec $n$ entier}
	\end{correctionbox}
	
	
	\subsection*{Exercice 17}
	On note $S_n^k$ le nombre de surjections de $[\![ 1, n ]\!]$ dans $[\![ 0, k ]\!]$
	\begin{enumerate}
		\item Préciser les valeurs de $S_n^1$ et $S_n^n$.
		\begin{correctionbox}
			Respectivement 1 et $n!$, je vous laisse le faire vous mêmes.
		\end{correctionbox}
		\item Montrer que $S_n^2=2n-2$
		\begin{correctionbox}
			$S_n^2 = \{1, 2\}^{[\![ 1, n ]\!]} \backslash \{i\mapsto1, i\mapsto2\}$, d'où le résultat.
		\end{correctionbox}
		\item Montrer que, pour $n\in \mathbb N$, $k\leqslant n$, $\displaystyle\sum_{j=1}^n \binom j k S_n^j = k^n$
		\begin{correctionbox}
			On caractérise une application par la restriction maximale sur laquelle elle est surjective (de taille $j$). Je vous le rédige proprement plus tard.
		\end{correctionbox}
		\item Montrer que $S_n^j=j (S_{n-1}^j + S_{n-1}^{j-1})$ \\
		flemme.
		\item mdrr je code pas en python moi
		\item plus tard
		\item En déduire que $\displaystyle \sum^n_{j=1} (-1)^{n-k} k^n \binom k n = n!$
		\begin{correctionbox}
			Se déduit de la question suivante, pour $j=n$.
		\end{correctionbox}
		\item Montrer que $\displaystyle \sum^n_{j=1} (-1)^{n-k} k^n \binom k n = n!$
		\begin{correctionbox}
			On considère l'application $P(X) \mapsto P(X+1)$. Elle est évidemment linéaire, et sa matrice dans la base canonique est $M=\left(\binom {i-1} {j-1} \right)$. \\
			Ainsi, l'inverse de cette matrice est la matrice associée à $P(X) \mapsto P(X-1)$, ie $M^{-1}=\left(  (-1)^{i+j}\binom {i-1} {j-1} \right)$. \\
			Or, d'après la question 2., $M \cdot {}^T (S^1_n \ \dots \ S^n_n) = {}^T (1^n \ \dots \ n^n)$, d'où, en composant par $M^{-1}$ à gauche: $M^{-1} \cdot {}^T (1^n \ \dots \ n^n) = {}^T (S^1_n \ \dots \ S^n_n)$, d'où la formule de l'énoncé.
		\end{correctionbox}
	\end{enumerate}
	
\end{document}
